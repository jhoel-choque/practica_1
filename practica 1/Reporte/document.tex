\documentclass[]{report}
\usepackage[spanish]{babel}

% Title Page
\title{Práctica 3}
\author{}


\begin{document}


\begin{abstract}
	
\end{abstract}
\section{Introducción}
	El problema dek agebte viajero (TSP, Travelling salesman problem) es uno de los problemas de optimización combinatoria más conocidos y estudiados en la teoría de la optimización y la investigación de operaciones.\\
	Este problema se plantea de la siguiente manera: Dado un conjunto de ciudades y las distancias entre cada par de ciudades, el objetivo es encontrar la ruta más corta que visite cada ciudad exactamente una vez y regrese a la ciudad de origen. El problema busca minimizar la distancia total recorrida por el agente viajero mientras cumple con las restricciones mencionadas.
	
	El TSP se puede modelar y resolver utilizando conceptos y técnicas de teoría de grafos, en donde las ciudades y las distancias entre ellas se pueden representar como un grafo completo en el que cada ciudad es un nodo y cada arista entre dos nodos representa la distancia entre esas dos ciudades. Las ponderaciones de las aristas (distancias) son las que deben ser minimizadas en el TSP. Aqui entra otro concepto de la teoria de grafos, los ciclos Hamiltonianos.
	
	Un ciclo Hamiltoniano en un grafo es un ciclo que visita cada nodo exactamente una vez. En el TSP, la solución óptima es un ciclo Hamiltoniano que también es de longitud mínima, es decir, la ruta más corta que visita todas las ciudades una vez y regresa a la ciudad de inicio.
	
	La mayoría de los algoritmos de resolución del TSP utilizan estructuras de grafos y técnicas de teoría de grafos para buscar soluciones óptimas o aproximadas. En este ssentido, los métodos exactos y heurísticos a menudo explotan propiedades del grafo, como la búsqueda de árboles generadores mínimos (MST) o la búsqueda de rutas óptimas basadas en el algoritmo de Dijkstra.
	
	Debido al gran costo computacional del TSP, es que se buscan soluciones de optimizacion, algunos de los cuales tienen un enfoque en el comportamiento de la naturaleza, como los algoritmos geneticos, busqueda de cuervos, colonia de hormigas, etc.
	
	El objetivo de esta práctica es explorar cómo los conceptos de teoría de grafos pueden ser aplicados para resolver y evaluar el rendimiento de un algoritmo específico diseñado para abordar el TSP.
\section{Algoritmos}
	El algoritmo usado en esta práctica
	
\section{Implementación}

\end{document}          
